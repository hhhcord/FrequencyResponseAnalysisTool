\documentclass[a4paper,twocolumn]{jarticle}

% 余白の設定
\usepackage[top=2cm, bottom=2cm, left=1.5cm, right=1.5cm]{geometry}

% 数式
\usepackage{amsmath,amsfonts,mathtools}
\usepackage{bm}

% 画像
\usepackage[dvipdfmx]{graphicx}

\begin{document}

\section*{予測誤差法}

このセクションでは、予測誤差法という制御理論における重要な手法を学びます。

\[
  (qw)_{(t)}\coloneqq w_{(t-1)}
\]
ここで、\( qw \)は遅延オペレータを表し、時刻\( t \)における値を時刻\( t-1 \)における値として表現します。

\[
  G(q)=\frac{b_nq^n+b_{n-1}q^{n-1}+\cdots+b_0}{q^n+a_{n-1}q^{n-1}+\cdots+a_1q+a_0}
\]
\( G(q) \)は伝達関数を表し、システムの入出力関係を示します。分子はシステムの出力を、分母はシステムの入力を表しています。

\begin{align*}
  y_{(k+n)}&=a_{n-1}y_{(k+n-1)}+\cdots+a_1y_{(k+1)}+a_0y_{(k)}\\
  &+b_nu_{(k+n)}+\cdots+b_1u_{(k+1)}+b_0u_{(k)}+e_{(k)}
\end{align*}
この方程式は、時刻\( k+n \)における出力\( y \)が、過去の出力、入力、および誤差\( e \)の関数であることを示しています。

\[
  Y=X_1v_1+X_2v_2+e
\]
これは、システムの出力\( Y \)が二つの項\( X_1v_1 \)と\( X_2v_2 \)、及び誤差\( e \)の和であることを示しています。

以下の方程式は、これらのベクトルと行列の定義を示しています。

\begin{align*}
  Y&\coloneqq \begin{bmatrix}
    y_{(n)}&y_{(n+1)}&\cdots&y_{(N-1)}&y_{(N)}
  \end{bmatrix}^\top\\
  X_1&\coloneqq \begin{bmatrix}
    y_{(n-1)}&\cdots&y_{(1)}&y_{(0)}\\
    y_{(n)}&\cdots&y_{(2)}&y_{(1)}\\
    \vdots&&\vdots&\vdots\\
    y_{(N-2)}&\cdots&y_{(N-n)}&y_{(N-n-1)}\\
    y_{(N-1)}&\cdots&y_{(N-n+1)}&y_{(N-n)}
  \end{bmatrix}\\
  v_1&\coloneqq \begin{bmatrix}
    a_{n-1}&a_{n-2}&\cdots&a_0
  \end{bmatrix}^\top\\
  X_2&\coloneqq \begin{bmatrix}
    u_{(n)}&\cdots&u_{(1)}&u_{(0)}\\
    u_{(n+1)}&\cdots&u_{(2)}&u_{(1)}\\
    \vdots&&\vdots&\vdots\\
    u_{(N-1)}&\cdots&u_{(N-n)}&u_{(N-n-1)}\\
    u_{(N)}&\cdots&u_{(N-n+1)}&u_{(N-n)}
  \end{bmatrix}\\
  v_2&\coloneqq \begin{bmatrix}
    b_{n}&b_{n-1}&\cdots&b_0
  \end{bmatrix}^\top\\
  e&\coloneqq \begin{bmatrix}
    e_{(n)}&e_{(n+1)}&\cdots&e_{(N-1)}&e_{(N)}
  \end{bmatrix}^\top\\
\end{align*}

\[
  \mathrm{arg}~\min_v\sum_{k=n}^{N}e(k)^2=(X^\top X)^{-1}X^\top Y
\]
この方程式は、誤差の二乗和を最小化することで、最適なパラメータ\( v \)を見つける方法を示しています。

\begin{align*}
  v&\coloneqq \begin{bmatrix}
    v_1^\top&v_2^\top
  \end{bmatrix}^\top\\
  X&\coloneqq \begin{bmatrix}
    X_1&X_2
  \end{bmatrix}
\end{align*}

\end{document}
